documentclass[14pt, a4paper]{extreport}
\usepackage[a4paper, total={6in, 10in}]{geometry}
\usepackage[utf8]{inputenc}
\usepackage{mathtools}
\usepackage[russian]{babel}
\thispagestyle{empty}
\begin{document}
\textit{Если $a_n$ > $0$, n = $1$, $2$, ... , то суммой $\sum_{n = 1}^{\infty}a_n$ называется конечный или бесконечный предел $\displaystyle\lim_{n \to \infty}$ $(a_1 + ... + a_n)$.}

Этот предел, конечный или равный $+\infty$, всегда существует, так как последовательность \{\textit{$a_1$} + ... + \textit{$a_n$}\}, в силу условия \textit{$a_n$} > 0, возрастает. Таким образом, $$\sum_{n = 1}^{\infty}a_n =^{def} \lim_{n \to \infty} \sum_{k = 1}^{n}a_k.$$ Суммы бесконечного числа слагаемых будут подробно рассмотрены во главе III.\\
\textbf{Определение 5} \textit{Множество X, лежащее на числовой оси, называется множеством лебеговой меры нуль, если для любого $\varepsilon$ > 0 существет покрытие этого множества конечной или счетной системой интервалов, сумма длин которых меньше $\varepsilon$. }

\textbf{Пример.} Всякое конечное или счетное множество является множеством лебеговой меры нуль.


Пусть X = \{\textit{$x_n$}\} - конечное или счетное множество (индекс \textint(n) может быть либо любым натуральным числом, когда X - счетное множество, либо принимать только значения, не превосходящие некоторого фиксированного натурального числа, тогда множество X конечно). Зададим произвольно $\varepsilon$ > 0. Система итервалов \textit{$(x_n - \frac{\varepsilon}{2^{(n+2)}}, x_n + \frac{\varepsilon}{2^{(n+2)}})$, n = $1$, $2$, ... ,} очевидно, покрывает множество X, а сумма их длин меньше  $\varepsilon$:$$\displaystyle\sum_{n = 1}^{\infty}a_n\frac{\varepsilon}{2^{n + 1}} = \frac{\varepsilon}{4}\frac{1}{1 - 1 / 2} = \frac{\varepsilon}{2} < \varepsilon.$$

Например, множество всех рациональных чисел является множеством лебеговой меры нуль.

\textbf{Задача 16.}\small{ Построить пример несчетного множества лебеговой меры нуль.}\\
\normalsize\textbf{Теорема 10 (теорема Лебега).} \normalsize\textit{Для того чтобы ограниченная на отрезке функция была на нем интегрируема, необходимо и достаточно, чтобы множество ее точек разрыва было множеством лебеговой меры нуль.}\\
\large{Доказательство необходимости. } \normalsize{Пусть функция \textit{f} интегрируема на отрезке X = [\textit{a, b}] и X_0 - множество ее точек разрыва. Зададим произвольно $\varepsilon$ > 0. Согласно критерию интегрируемости Дюбуа-Реймона, для каждого \textit{n} = 1, 2, ...}


\end{document}
